\documentclass[compress,mathserif]{beamer}
\usepackage{beamerthemesplit_z}
\usepackage{bm,bbm,graphicx}
\usepackage[czech]{babel}
\usepackage{pgf,pgfarrows,pgfnodes,pgfautomata,pgfheaps}
\usepackage{yhmath,amsmath,amssymb,amsbsy}
\usepackage[utf8]{inputenc}
\usepackage[style=german]{csquotes}
\usepackage{psfrag}
\title[Příklady]{Příklady}
\author{Lenka Křivánková}
\institute{142474@mail.muni.cz}
\date{}
\usetemplatetocsection
%{\color{structure}\inserttocsection}
{\color{structure}\insertsection}


\begin{document}

\frame{\titlepage
\begin{center}
{Statistika I (BKMSTAI)\\[5mm] Brno}
%	{Přednáška Statistika I (BKMSTAI)\\[5mm] 23. říjen 2015, Brno}
\end{center}}


\frame {
  \frametitle{Příklad 1.}

Při statistickém šetření pojištěnců byly získány tyto výše pojistek (v Kč):
\begin{center}
\tabcolsep=3pt
\begin{tabular}{lrrrrrrrrrr}
Výše pojistky	&390&	410&	430&	450&	470&	490&	510&	530&	550&	570\\
Počet pojištěnců&	7&	10	&14	&22	&25	&12&	3	&3&	2&	2\\
\end{tabular}
\end{center}
\begin{itemize}
\item[a)]	Nakreslete graf četnostní funkce. \\[3mm]
\item[b)]	Zjistěte průměr, medián a modus výše pojistky. \\[3mm]
\item[c)]	Vypočtěte rozptyl, směrodatnou odchylku a koeficient variace výše pojistky.  
\end{itemize}

}


\frame {
  \frametitle{Příklad 2.}
Během prvního čtvrtletí došlo k růstu prodejů firmy o 3 \%, ve druhém čtvrtletí došlo k nárůstu o 8 \% a ve třetím čtvrtletí byl pokles 2 \%. Vypočítejte, jaká změna by musela být ve čtvrtém čtvrtletí, aby firma měla průměrný růst 2 \%.
\\[20mm]
\begin{flushright}
\scriptsize Odpověď: Pokles o 0,7 \%.
\end{flushright}
}


\frame {
  \frametitle{Příklad 3.}
Počet různých druhů zboží, které zákazník nakoupí při jedné návštěvě obchodu, je náhodná veličina X. Dlouhodobým sledováním bylo zjištěno, že X nabývá hodnot 0; 1; 2; 3; 4 s~pravděpodobnostmi 0,25; 0,55; 0,11; 0,07 a 0,02.\\[1cm]
\begin{itemize}
\item[a)]	Najděte distribuční funkci náhodné veličiny X a nakreslete její graf. \\[0.5cm]
\item[b)]	Vypočtěte střední hodnotu náhodné veličiny X. \\[0.5cm]
\item[c)]	Vypočtěte rozptyl náhodné veličiny X. 
\end{itemize}

}



\frame {
  \frametitle{Ukázka průběžné písemné zkoušky na denním studiu -- Příklad 1.}
Jev $A$ spočívá v tom, že při hodu kostkou padne číslo větší než 2 a jev $B$ v tom, že padne sudé číslo. Určete, co uvedené jevy znamenají a vypište možné výsledky příznivé nastoupení uvedených jevů.\\[0.5cm]
\begin{itemize}
	\item[a)] $A\cap B$
		\item[b)] $A \backslash B$
			\item[c)] Rozhodněte, zda jsou jevy $A$ a $B$ stochasticky nezávislé. Zdůvodněte.
\end{itemize}

}

\frame {
  \frametitle{Ukázka průběžné písemné zkoušky na denním studiu -- Příklad 2.}
Rozhodněte, zda je následující tvrzení pravdivá. Svou odpověď zdůvodněte. 
\begin{itemize}
	\item[a)] Pro libovolné neslučitelné jevy $A$ a $B$ platí
	$$
	P(A)\cup P(B)= P(A+B)
	$$
		\item[b)] Koeficient variace je podíl směrodatné odchylky a mediánu.
			\end{itemize}
}

\frame {
  \frametitle{Ukázka průběžné písemné zkoušky na denním studiu -- Příklad 3.}
Vypočtěte $\iint\limits_N \frac{1}{3}dydx$, kde množina $N$ je tvořena trojúhelníkem s~vrcholy v bodech $[0,0]$; $[1,1]$ a $[0,1]$.

}


\frame {
  \frametitle{Ukázka průběžné písemné zkoušky na denním studiu -- Příklad 4.}
Tři výrobci dodávají žárovky do obchodu. První dodává 45 \%, druhý 40 \% a třetí 15 \% celkového množství. První dodavatel má 70 \% standardních žárovek, druhý 80 \% a třetí 81 \%. Určete pravděpodobnost, že si zákazník koupí standardní žárovku. 
}

\end{document}