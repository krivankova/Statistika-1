\documentclass[compress,mathserif]{beamer}
\usepackage{beamerthemesplit_z}
\usepackage{bm,bbm,graphicx}
\usepackage[czech]{babel}
\usepackage{pgf,pgfarrows,pgfnodes,pgfautomata,pgfheaps}
\usepackage{yhmath,amsmath,amssymb,amsbsy}
\usepackage[utf8]{inputenc}
\usepackage[style=german]{csquotes}
\hypersetup{%
  pdftitle={Zákon velkých čísel, centrální limitní věta},%
  pdfauthor={David Hampel}}

\title[Zákon velkých čísel, centrální limitní věta]{Zákon velkých čísel, centrální limitní věta}
\author{Lenka Křivánková}
\institute{142474@mail.muni.cz}
\date{}

\usetemplatetocsection
%{\color{structure}\inserttocsection}
{\color{structure}\insertsection}


\begin{document}

\frame{\titlepage
\begin{center}

	{Přednáška Statistika I (BKMSTAI)\\[5mm] 4. listopad 2016, Brno}
\end{center}}


\frame {
  \frametitle{Motivace}
  Při velkém počtu pozorování  se ukazuje, že
  \begin{itemize}
  \item {\bf empirické charakteristiky se blíží teoretickým charakteristikám},
	\item odhad sledovaných veličin se zpřesňuje přímo úměrně velikosti výběru,
  \item určité transformované veličiny mají téměř normální rozdělení,
	\item je možno použít přibližných vzorců pro pozorování, jehož rozdělení není tabelováno.
	\end{itemize}
  }
  
\frame {
  \frametitle{Konvergence náhodných veličin}
  Nechť $X_1, X_2, \dots$ je posloupnost náhodných veličin s distribučními funkcemi $\Phi_1(x_1), \Phi_2(x_2), \dots$ a $X$ náhodná veličina s distribuční funkcí $\Phi(x)$. Řekneme, že posloupnost $X_1, X_2, \dots$ konverguje k $X$
  \begin{itemize}
	\item {\bf jistě}, právě když pro všechna $\omega\in\Omega$ platí
	$$
	\lim_{n\rightarrow\infty}X_n(\omega) = X(\omega),
	$$
	\item {\bf podle pravděpodobnosti}, právě když pro všechna $\epsilon >0$ platí
	$$
	\lim_{n\rightarrow\infty}P(|X_n-X|>\epsilon) = 0,
	$$
	\item {\bf v distribuci}, právě když pro všechna $x\in R$ platí
	$$
	\lim_{n\rightarrow\infty}\Phi_n(x) = \Phi(x).
	$$
\end{itemize} 
  }
  
\frame {
  \frametitle{Čebyševova věta}
  Nechť $X_1, \dots, X_n$ jsou nekorelované náhodné veličiny, jejichž střední hodnoty splňují vztah
  $$
  \lim_{n\rightarrow \infty} \frac{1}{n}\sum_{i=1}^n E(X_i) = \mu
  $$
  a rozptyly jsou shora ohraničené týmž číslem $\delta$. Pak posloupnost aritmetických průměrů
  $$
  \left\{ X_1, \frac{1}{2} \sum_{i=1}^2 X_i, \dots, \frac{1}{n} \sum_{i=1}^n X_i, \dots \right\}
  $$
  
  konverguje podle pravděpodobnosti k číslu $\mu$.
  }
  
  
\frame {
  \frametitle{Bernoulliova věta}
  Nechť náhodná veličina $Y_n$ udává počet úspěchů v posloupnosti $n$ nezávislých opakovaných pokusů, přičemž úspěch nastává v~každém pokusu s pravděpodobností $\theta$, $0<\theta<1$. Pak posloupnost relativních četností
  $$
  \{ Y_1, Y_2/2, \dots, Y_n/n, \dots\}
  $$
  konverguje podle pravděpodobnosti k pravděpodobnosti úspěchu $\theta$.
  
  }



\frame {
  \frametitle{Čebyševova nerovnost}
  Pro jakoukoliv náhodnou veličinu $X$, která má střední hodnotu $E(X)$ a rozptyl $D(X)$, je pravděpodobnost toho, že absolutní odchylka $|X-E(X)|$ nabude hodnoty menší než libovolné $\epsilon > 0$
  $$
  P(|X-E(X)|<\epsilon) \geq 1 - \frac{D(X)}{\epsilon^2}.
  $$
  Této nerovnosti můžeme využít pro odhad uvedené pravděpodobnosti, neznáme-li rozdělení dané náhodné veličiny.
  
  }
  
  \frame {
  \frametitle{Čebyševova nerovnost -- příklad}
  Víme, že náhodná veličina $X$ má střední hodnotu 3 a rozptyl 4. Máme odhadnout pravděpodobnost, že veličina $X$ nabude hodnoty z intervalu $[-2, 8]$.
  
  Hledáme tedy pravděpodobnost
  $$ 
  P(-2<X<8) = P(|X-E(X)|<5),
  $$
  která je dle Čebyševovy nerovnosti 
   $$ 
  P(|X-E(X)|<5) \geq 1-\frac{4}{25} = 0.84.
  $$
  }
  
  
\frame {
  \frametitle{Lindberg-Lévyova centrální limitní věta}
  Nechť $X_1, X_2, \dots$ je posloupnost stochasticky nezávislých náhodných veličin se stejným rozdělením, $E(X_i) = \mu$ a $D(X_i)= \sigma^2$ pro $i=1, 2, \dots$. Pak posloupnost standardizovaných součtů
  $$
  \left\{
  \frac{\sum_{i=1}^n X_i - n\mu}{\sigma\sqrt{n}}
  \right\}_{n=1}^\infty
  $$
  konverguje v distribuci ke {\bf standardizované normální náhodné veličině}, tj. pro každé $x\in R$ platí
  $$
  \lim_{n\rightarrow\infty}P\left( \frac{\sum_{i=1}^n X_i - n\mu}{\sigma\sqrt{n}} \leq x \right) = \int_{-\infty}^x \frac{1}{\sqrt{2\pi}}e^{-t^2/2}dt.
  $$
  }
  
  
  \frame {
  \frametitle{Moivre-Laplaceova integrální věta}
  Nechť $Y_1, Y_2, \dots$ je posloupnost stochasticky nezávislých náhodných veličin, $Y_i \sim Bi(n, \theta)$ pro $i=1, 2, \dots$. Pak  posloupnost standardizovaných náhodných veličin
  $$
  \left\{
  \frac{Y_n  - n\theta}{\sqrt{n\theta(1-\theta)}}
  \right\}_{n=1}^\infty
  $$
  konverguje v distribuci k náhodné veličině $U \sim N(0,1)$, tj. pro každé $x\in R$ platí
  $$
  \lim_{n\rightarrow\infty}P\left( \frac{Y_n  - n\theta}{\sqrt{n\theta(1-\theta)}} \leq x \right) = \int_{-\infty}^x \frac{1}{\sqrt{2\pi}}e^{-t^2/2}dt.
  $$
  
  
  }
  
  
  
  
  \frame {
  \frametitle{Moivre-Laplaceova integrální věta}
  Na základě této věty se používá přibližného vzorce, který nahrazuje pracný výpočet distribuční funkce binomického rozdělení tabelovanou hodnotou distribuční funkce standardizovaného normálního rozdělení:
  
  $$
  P(Y_n\leq y) = \sum_{t=0}^y\binom{n}{t}(1-\theta)^{n-t}\theta^t =
  $$
  $$
  = P\left( \frac{Y_n  - n\theta}{\sqrt{n\theta(1-\theta)}} \leq \frac{y  - n\theta}{\sqrt{n\theta(1-\theta)}}
  \right) \approx \Phi\left( \frac{y  - n\theta}{\sqrt{n\theta(1-\theta)}} \right) .
  $$
  Aproximace se považuje za vyhovující, jsou-li splněny podmínky
  $$
  n\theta(1-\theta) > 9 \quad \mathrm{a} \quad \frac{1}{n+1}<\theta<\frac{n}{n+1}.
  $$
  }
  
  \frame {
  \frametitle{Moivre-Laplaceova integrální věta -- příklad 1}
 Pravděpodobnost, že určitý typ výrobku má výrobní vadu, je 0.05. Jaká je pravděpodobnost, že ze série 1000 výrobků bude mít výrobní vadu nejvýše 70?
 
 Označíme $Y_n$ náhodnou veličinu, která udává počet vadných výrobků ze série $n$ výrobků. Zřejmě je
 $$ Y_n \sim Bi(n,0.05), \quad n\theta(1-\theta)= 1000 \cdot 0.05 \cdot 0.95 = 47.5>9
 $$ 
 $$
 \mathrm{a}\quad \frac{1}{n+1}= \frac{1}{1001} <0.05 <\frac{n}{n+1}=\frac{1000}{1001}.
 $$
 Spočteme\small
 $$
 P(Y_{1000}\leq 70)= P\left( \frac{Y_{1000}  - 1000\cdot 0.05}{\sqrt{1000\cdot  0.05(1-0.05)}} \leq \frac{70  - 1000 \cdot 0.05}{\sqrt{1000 \cdot 0.05(1-0.05)}}
  \right) \approx
  $$
  $$
   \approx \Phi\left( \frac{70  - 1000\cdot 0.05}{\sqrt{1000\cdot  0.05(1-0.05)}} \right) =\Phi\left( \frac{20}{\sqrt{47.5}}\right)=\Phi(2.90)=0.99813.
  $$
 \normalsize
  }
  
  
   \frame {
  \frametitle{Moivre-Laplaceova integrální věta -- příklad 2}
 Pravděpodobnost, že výrobek má 1. jakost, je $\theta = 0.9$. Kolik výrobků je třeba zkontrolovat, aby s pravděpodobností aspoň 0.99 bylo zaručeno, že rozdíl relativní četnosti počtu výrobků 1. jakosti a pravděpodobnosti $\theta = 0.9$ byl v absolutní hodnotě menší než 0.03?
 
 Hledáme pravděpodobnost 
 $$
 P\left(\left|\frac{X}{n}-0.9\right|<0.03\right)\geq 0.99
 $$ 
 Výpočet:
 \small
 $$
 0.99 \leq P(0.87n<X<0.93n) =
 $$
 $$
 =P(\frac{0.87n-0.9n}{\sqrt{0.09n}}<\frac{X-0.9n}{\sqrt{0.09n}}<\frac{0.93n-0.9n}{\sqrt{0.09n}})=
 $$
 $$
 = P(-0.1\sqrt{n}< \frac{X-0.9n}{\sqrt{0.09n}}  <0.1\sqrt{n})\approx 
 $$
 $$
 \approx \Phi(0.1\sqrt{n}) - \Phi(-0.1\sqrt{n}) = 2\Phi(0.1\sqrt{n})-1
 $$
 
   } 
  
 \frame {
  \frametitle{Moivre-Laplaceova integrální věta -- příklad 2}
  První a poslední člen této nerovnosti je
  \begin{eqnarray*}
  0.99 &\leq& 2\Phi(0.1\sqrt{n})-1\\
  0.995 &\leq& \Phi(0.1\sqrt{n}) \quad \quad \quad /\Phi^{-1}\\
  2.57583&\leq&0.1\sqrt{n}\\
  664.76&\leq& n\\
  \end{eqnarray*}
  Abychom dosáhli požadované pravděpodobnosti, musíme zkontrolovat alespoň 665 výrobků. 
  
  } 
  
  
  

\end{document}