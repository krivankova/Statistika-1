\documentclass[compress,mathserif]{beamer}
\usepackage{beamerthemesplit_z}
\usepackage{bm,bbm,graphicx}
\usepackage[czech]{babel}
\usepackage{pgf,pgfarrows,pgfnodes,pgfautomata,pgfheaps}
\usepackage{yhmath,amsmath,amssymb,amsbsy}
\usepackage[utf8]{inputenc}
\usepackage[style=german]{csquotes}
\usepackage{url}

\title[Úvodní informace]{Úvodní informace}
\author{Lenka Křivánková}
\institute{142474@mail.muni.cz}
\date{}

\usetemplatetocsection
%{\color{structure}\inserttocsection}
{\color{structure}\insertsection}


\begin{document}

\frame{\titlepage
\begin{center}

	{Přednáška Statistika 1 (BKMSTA1)\\[5mm] 8. říjen 2016, Brno}
\end{center}
}%\rule[-1.9cm]{5cm}{0cm}


\frame{
\frametitle{O mě}
\begin{itemize}
	\item Na ESF MU \textbf{externí} vyučující (přednášky BKMSTA1, BKMSTA2).\\[4mm]
%	\item Obor činnosti: Pravděpodobnost a matematická statistika, aplikovaná statistika, ekonometrie.\\[4mm]
	\item Hlavní zaměstnání: 
	\begin{itemize}
	   \item Risk Analyst ve Sberbank CZ\\
		 Reporting, Analytics and Local Models \\[2mm]
	   \item PhD studium na PřF MU\\
		 (Stochastické metody ve finanční matematice)\\[5mm]
  \end{itemize}
	\item {\bf Kontakt přes e-mail v ISu, 142474@mail.muni.cz.}
\end{itemize}}



\frame{
\frametitle{Přednášky}
\begin{itemize}
	\item Sobota 8. 10. 8:30--11:50, P102 \& P106  \\[4mm]
	%\item Pátek 23. 10. 16:20--19:35, P101\\[4mm]
	\item Pátek 21. 10. 12:50--16:15, P101\\[4mm]
	%\item Sobota 14. 11. 8:30--11:50, P101 \\[1cm]
	\item  Pátek 4. 11. 12:50--16:15, P101 \\[1cm]
	\item Očekávám, že v období mezi přednáškami mi e-mailem pošlete své nejasnosti, na další přednášce upřesním, dovysvětlím.
	\end{itemize}
}

\frame{
\frametitle{Práce opravená tutorem (POT)}
\begin{itemize}
	\item termín 25.--27. 11. 2016\\[5mm]
	\item možnost jedné opravy
\end{itemize}



}


\frame{
\frametitle{Zkoušky}


\begin{center}
			\hspace*{-0.4cm}
		\begin{tabular}{ccccc}
		\bfseries{Termín}&\bfseries{Den}&\bfseries{Čas}&\bfseries{Místnost}&\bfseries{Kapacita}\\ \hline
		Ř& Pá 16. 12. 2016& 14:00--15:30&  P101&90\\
		%Ř& So 19. 12. 2014& 09:00--10:30&  P101&90\\
		%Ř+O& So 14. 12. 2013& 09:00--10:30& P1&90\\
		%Ř+O& So 10. 01. 2015& 09:00--10:30& P101&90\\
		Ř+O& Ne 22. 01. 2015& 17:00--18:30& P101&90\\
		O& Ne 05. 02. 2015& 17:00--18:30& P101&90\\
		%Ř+O& So 13. 02. 2015& 09:00--10:30& P101&90\\
		\end{tabular}
\end{center}
	\begin{itemize}
  %\item V případě zájmu (alespoň 20 studentů) vypíšu předtermín na prosinec 2012 (předběžně 15. 12.). \\[4mm]
	\item Po zkouškovém období \textbf{nebudou} další termíny.\\[4mm]
	\item {\bf Na Statistiku I navazuje Statistika II, kterou nebude možné bez úspěšného vykonání zkoušky ze Statistiky~I zapsat.}
\end{itemize}
}


\frame{\frametitle{Obsah přednášek}
\begin{itemize}
	\item {\bf Popisná statistika}\\[2mm]
	\begin{itemize}
	\item Základní, výběrový a datový soubor\\[1mm]
	\item Bodové a intervalové rozdělení četností\\[1mm]
	\item Číselné charakteristiky znaků\\[1mm]
	\item Regresní přímka\\[0.7cm]
	\end{itemize}
	\item {\bf Pravděpodobnost}\\[2mm]
	\begin{itemize}
	\item Jev a jeho pravděpodobnost\\[1mm]
	\item Stochasticky nezávislé jevy a podmíněná pravděpodobnost\\[1mm]
	\item Náhodná veličina a její distribuční funkce\\[1mm]
	\item Vybraná rozdělení diskrétních a spojitých náhodných veličin\\[1mm]
	\item Číselné charakteristiky náhodných veličin\\[1mm]
	\item Zákon velkých čísel a centrální limitní věta
	\end{itemize}
%	\item {]bf Základy matematic{é staôistiky}
%	\begin{itumize}
%	\itĺm Záklafní pojmy matematicëé statistik9
%	]item Bodové a intervalové odhady parametrů a parametrických funkcíŠ%	\iuem Úvod do testování hypotéz a dgsty o parametrdch normálního rozdě,e~í
%	Tend{itemize}
\end{itemize}}


\frame{
\frametitle{Požadavky pro úspěšné ukončení}
\begin{itemize}
	\item Podrobněji ve Studijních materiálech v ISu, složka Organizační pokyny.\\[9mm]
	\item[1.] Úspěšné vypracování POTu.\\[2mm]
	\begin{itemize}
	\item POT je ve formě testu na počítači (bez dohledu). \\[2mm]
	\item Je vhodné ověřit si \textbf{funkčnost} na cvičných testech. \\[2mm]
	\item 11 otázek, 60 minut, \textbf{1 bod} za správnou odpověď, \textbf{-0,5 bodu} za špatnou odpověď, \textbf{0 bodů} za chybějící odpověď. \\[2mm]
	\item Je třeba získat minimálně \textbf{6 bodů}.
	\end{itemize}
 \end{itemize}}

\frame{
\frametitle{Požadavky pro úspěšné ukončení}
\begin{itemize}
  \item[2.] Úspěšné složení zkoušky.\\[2mm]
\begin{itemize}
	\item Při zkoušce je možno používat studijní materiály a kalkulačku. \\[2mm]
	\item Není možné používat mobilní telefon ani jiné komunikačně zařízení (ani jako hodinky či kalkulačku).\\[2mm]
	\item Zkouška je písemná, trvá 90 minut, skládá se z několika okruhů rozdělených do podotázek. \\[2mm]
	\item Je třeba získat minimálně 6 bodů z celkových 12. Bodování viz POT.
	\end{itemize}
\end{itemize}}


\frame{
Jakékoli opisování, zaznamenávání nebo vynášení testů, používání nedovolených pomůcek  jakož i komunikačních prostředků nebo jiné narušování objektivity zkoušky (zápočtu) bude považováno za nesplnění podmínek k ukončení předmětu a za hrubé porušení studijních předpisů. Následkem toho uzavře vyučující zkoušku (zápočet) hodnocením v ISu známkou "F" a děkan zahájí disciplinární řízení, jehož výsledkem může být až ukončení studia.}



\frame{
\frametitle{Literatura}
\begin{itemize}
\item Doporučená studijní literatura: Budíková, Králová, Maroš: Průvodce základními statistickými metodami. Grada 2010 (dotisk 2011). \url{http://www.grada.cz/pruvodce-zakladnimi-statistickymi-metodami_6069/kniha/katalog/}
\item {\ Ve studijních materiálech je aktuální inovovaná verze DSO.}\\[5mm]
\item Původní DSO: Budíková, Marie (2004, popř. novější vydání). DSO Statistika. Masarykova univerzita, Brno\\[2mm]
\begin{itemize}
	\item Dostupná v tištěné podobě.\\[2mm]
	\item {\bf Pro přípravu na zkoušku není dostatečná.}\\[5mm]
\end{itemize}
\item Další literatura není vyžadována, odkazy na vhodné doplnění jsou uvedeny.
	\end{itemize}}


\frame{
\frametitle{Software Statistica}
\begin{itemize}
	\item Silný výpočetní nástroj, mnoho statistických funkcí.\\[4mm]
	\item Práce podobná jako v prostředí MS Excelu. \\[4mm]
	\item Pro studenty MU multilicence.\\[4mm]
	\item Možno stáhnout ze stránek https://inet.muni.cz/app/soft/licence\\[4mm]
	\item Další informace na \\[3mm] 
		\hspace*{-1.5cm}{https://wiki.ics.muni.cz/doku.php?id=softwarove\_licence:licence\#statistica}
\end{itemize}}


\frame{
\frametitle{Specifika předmětu}
\begin{itemize}
	\item Statistika a pravděpodobnost vyžaduje poněkud jiné myšlení než ostatní předměty v rámci ESF, včetně Matematiky.\\[4mm]
	\item Podobně jako u matematiky nutno přemýšlet, nestačí se mechanicky učit nazpaměť.\\[4mm]
	\item Distanční forma Statistiky jednodušší, méně obsažná, nicméně chybí cvičení a samostudium je obecně náročnější.\\[4mm]
	\item Fakt, že ke zkoušce mohou být veškeré materiály, je zrádný. V~omezeném čase se v žádném případě nestihnete naučit problematiku přímo na zkoušce. Je třeba alespoň taková příprava, abyste věděli, kde přesně hledat, a rozuměli postupu výpočtu všech vzorových příkladů.
\end{itemize}}


\end{document}