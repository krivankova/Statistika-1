\documentclass[compress,mathserif]{beamer}
\usepackage{beamerthemesplit_z}
\usepackage{bm,bbm,graphicx}
\usepackage[czech]{babel}
\usepackage{pgf,pgfarrows,pgfnodes,pgfautomata,pgfheaps}
\usepackage{yhmath,amsmath,amssymb,amsbsy}
\usepackage[utf8]{inputenc}
\usepackage[style=german]{csquotes}

\title[Základy pravděpodobnosti]{Základy pravděpodobnosti}
\author{Lenka Křivánková}
\institute{142474@mail.muni.cz}
\date{}

\usetemplatetocsection
%{\color{structure}\inserttocsection}
{\color{structure}\insertsection}


\begin{document}

\frame{\titlepage
\begin{center}

	{Přednáška Statistika 1 (BKMSTAI)\\[5mm] 21. říjen 2016, Brno}
\end{center}}


\frame {
  \frametitle{Motivace}
  \begin{itemize}
\item Teorie pravděpodobnosti se snaží matematicky popsat činnosti (\enquote{pokusy}), jejichž výsledek není předem jistý. 

\item Matematická statistika (odhady parametrů a testování hypotéz o nich) je založena na výsledcích teorie pravděpodobnosti.

\item Ačkoliv není teorie pravděpodobnosti mnohdy přímo aplikovatelná, pro celkové pochopení statistiky je její znalost nutná.

\end{itemize}
}


\frame {
  \frametitle{Pokus}
  \begin{itemize}
\item {\em Pokusem} rozumíme jednorázové uskutečnění konstantně vymezeného souboru de\-fi\-nič\-ních podmínek. Předpokládáme, že pokus můžeme mno\-ho\-ná\-sob\-ně nezávisle opakovat za dodržení definičních podmínek (ostatní podmínky se mohou měnit, proto různá opakování pokusu mohou vést k~různým vý\-sled\-kům). 

\item {\em Deterministickým pokusem} nazýváme takový pokus, jehož každé opakování vede k~jedinému možnému výsledku. (Např. zahřívání vody na $100\,^\circ\mathrm{C}$ při atmosférickém tlaku $1015\,\mathrm{hPa}$ vede k~varu vody.)

\item {\em Náhodným pokusem} nazýváme takový pokus, jehož každé opakování vede k~právě jednomu z~více \emph{možných výsledků}, které jsou vzájemně neslučitelné. 
\end{itemize}
  }
  
  

\frame {
  \frametitle{Náhodný jev a jeho pravděpodobnost}
  \begin{itemize}
\item  Neprázdnou množinu možných výsledků náhodného pokusu značíme $\Omega$ a nazýváme ji {\bf základní prostor}. Možné výsledky značíme $\omega_t$, $t\in T$ kde $T$ je indexová množina.
  \begin{itemize}
	\item Příklad: hážeme jedenkrát jednou pravidelnou šestistěnnou kostkou. Základním prostorem je množina $\{1, 2, 3, 4, 5, 6\}$.
\end{itemize}
  
\item    {\bf Jevem} nazveme kombinace prvků základního prostoru, které můžeme přesně popsat.\\ Jevem je:
\begin{itemize}
	\item Padne pětka.
	\item Padne jednička nebo dvojka.
	\item Padne liché číslo.
	\item Padne číslo vyšší nebo rovné třem. 
\end{itemize}
Jevem není:
\begin{itemize}
  \item Dva.
	\item Padne vysoké číslo.
\end{itemize}

  
  
\end{itemize}
  }


  \frame {
  \frametitle{Náhodný jev a jeho pravděpodobnost}
  
  \begin{itemize}
	\item Zvláštním případem je {\bf jev nemožný} -- sice jsme přesně popsali možný výsledek pokusu, ale tento výsledek nemůže nastat (není kombinací prvků základního prostoru).
  \begin{itemize}
	\item Padne číslo devět.
\end{itemize}

\item Označení používaná v souvislosti s jevy\\[5mm]

\hspace*{-1.6cm}
\begin{tabular}{rp{3.5cm}|rp{3.5cm}}
$\Omega$&jev jistý&$\emptyset$&jev nemožný\\ \hline
$\bigcap\limits_{i\in I}A_i$&společné nastoupení jevů $A_i$, $i\in I$&$\bigcup\limits_{i\in I}A_i$&nastoupení alespoň jednoho z jevů $A_i$, $i\in I$\\ \hline
\rule[-0.6cm]{0cm}{1cm}$\bar{A}_i$, $A'_i$&opačný jev k jevu $A_i$&$A_1\backslash A_2$&nastoupení jevu $A_1$ za nenastoupení jevu $A_2$\\ \hline
$A_1\subseteq A_2$&jev $A_1$ má za důsledek jev $A_2$&$A_1\cap A_2 = \emptyset$&jevy $A_1$ a $A_2$ jsou neslučitelné\\ 
\end{tabular}
\end{itemize}
  
 
  }



  \frame {
  \frametitle{Náhodný jev a jeho pravděpodobnost}
  
  \begin{itemize}
	\item {\bf Pravděpodobností} rozumíme funkci, která každému jevu přiřadí reálné číslo tak, aby platily následující podmínky:
  \begin{itemize}
	\item nezápornost $$P(A)\geq 0 \quad \forall A,$$
	\item spočetná aditivita $\forall i, j, i\neq j$ $P(A_i \cap A_j) = 0$ $\Rightarrow$ 
	$$
	P(\bigcup_{i=1}^\infty A_i) = \sum_{i=1}^\infty P(A_i),
	$$
	\item normovanost $$P(\Omega) = 1.$$
\end{itemize}

\end{itemize}
  
  
  

  }





\frame {
  \frametitle{Klasická pravděpodobnost}
  Označme $m(\Omega)$ počet všech možných výsledků a $m(A)$ počet výsledků příznivých nastoupení jevu $A$. Pak funkci
  $$P(A)=\frac{m(A)}{m(\Omega)}$$
  nazveme {\bf klasickou pravděpodobností}. Předpokladem použití klasické pravděpodobnosti je, aby každý výsledek pokusu nastal se stejnou pravděpodobností.\\[0.5cm]
  
  Příklad: hážeme jedenkrát jednou pravidelnou šestistěnnou kostkou. Díky pravidelnosti kostky je padnutí každého z čísel stejně pravděpodobné. Chceme spočítat pravděpodobnost jevu $A$: padne sudé číslo. Máme $m(\Omega)=6$, $m(A)=3$ a $P(A)=\frac{m(A)}{m(\Omega)}=\frac{3}{6}=\frac{1}{2}$.
  }

\frame {
  \frametitle{Klasická pravděpodobnost -- příklad}
  \small
Jaká je pravděpodobnost, že při současném hodu šesti kostkami padne
\begin{itemize}
	\item na každé kostce jiné číslo 
	$$P(A)=\frac{6!}{6^6}=\frac{720}{46656} = 0.01543$$
		\item právě šest šestek 
	$$P(B)=\frac{1}{6^6}=\frac{1}{46656} = 0.0000021$$
	\item právě pět šestek 
	$$P(C)=\frac{6\cdot 5}{6^6}=\frac{30}{46656} = 0.000643$$
	\item právě čtyři šestky
	$$P(D)=\frac{\binom{6}{2}\cdot 5 \cdot 5}{6^6}=\frac{375}{46656} = 0.008037$$
	\item samá sudá čísla
	$$P(E)=\frac{3^6}{6^6}=\frac{729}{46656} = 0.015625$$
\end{itemize}\normalsize
 }

\frame {
  \frametitle{Klasická pravděpodobnost -- příklad}
Mezi $N$ výrobky je $M$ zmetků. Náhodně bez vracení vybereme $n$ výrobků. Jaká je pravděpodobnost, že vybereme právě $k$ zmetků?\\[4mm]

Základní prostor $\Omega$ je tvořen všemi neuspořádanými $n$-ticemi vytvořenými z~$N$ prvků. Tedy $m(\Omega)=\binom{N}{n}$. Jev $A$ spočívá v~tom, že vybereme právě $k$ zmetků z~$M$ zmetků (ty lze vybrat $\binom{M}{k}$ způsoby) a výběr doplníme $n-k$ kvalitními výrobky vybranými z~$N-M$ kvalitních výrobků (tento výběr lze provést $\binom{N-M}{n-k}$ způsoby). Podle kombinatorického pravidla součinu do\-stá\-vá\-me 
$$m(A)=\binom{M}{k}\binom{N-M}{n-k}, \quad\mbox{tedy}\quad P(A)=\frac{m(A)}{m(\Omega)}=\frac{\binom{M}{k}\binom{N-M}{n-k}}{\binom{N}{n}}.$$
}

\frame {
  \frametitle{Podmíněná pravděpodobnost}
  Nechť $H$ je jev s nenulovou pravděpodobností. Pak definujeme {\bf podmíněnou pravděpodobnost} vzorcem 
  $$
  P(A|H)=\frac{P(A\cap H)}{P(H)}.
  $$
  Interpretace této pravděpodobnosti může být následující: víme, že jev $H$ již nastal, a ptáme se na pravděpodobnost, s jakou za této podmínky nastane ještě jev $A$.\\[1cm]
  
  Důležitou aplikací podmíněné pravděpodobnosti je věta o násobení pravděpodobností:
  $$
  P(\bigcap\limits_{i=1}^n A_i) = P(A_1)\cdot P(A_2|A_1)\cdot P(A_3|A_1\cap A_2)\cdot \dots \cdot P(A_n|A_1\cap\dots\cap A_{n-1})
  $$
  
   
  }
  
  \frame {
  \frametitle{Podmíněná pravděpodobnost -- příklad}
  

  Z 5 výrobků, mezi nimiž jsou 3 zmetky, vybíráme bez vracení po jednom výrobku. Označíme $A_1$: první vybraný výrobek byl kvalitní,  $A_2$: druhý vybraný výrobek byl zmetek, $A_3$: třetí vybraný výrobek byl zmetek. Hledáme $P(A_1\cap A_2 \cap A_3)$.
  
  $$P(A_1\cap A_2 \cap A_3)=P(A_1)\cdot P(A_2|A_1)\cdot P(A_3|A_1\cap A_2) = \frac{2}{5}\cdot\frac{3}{4}\cdot\frac{2}{3} = 0.2$$
   
  
  }
  
\frame {
  \frametitle{Formule úplné pravděpodobnosti}
  Nechť je dán rozklad $\{H_i, i\in I\}$ základního prostoru  na nejvýše spočetně mnoho jevů $H_i$ o nenulových pravděpodobnostech $P(H_i)$. Říkáme, že je dán {\bf úplný systém hypotéz}.\\[0.5cm] 
  Potom pro libovolný jev $A$ platí {\bf formule úplné pravděpodobnosti}
  $$
  P(A) = \sum_{i\in I} P(H_i)\cdot P(A|H_i).
  $$
  
  }
  
  
  
  \frame {
  \frametitle{Formule úplné pravděpodobnosti -- příklad}
  V první sadě výrobků je 12 výrobků, z toho 1 zmetek. Ve druhé sadě výrobků je 10 výrobků, z toho 1 zmetek. Náhodně zvolený výrobek jsme přemístili z první sady do druhé. Poté jsme ze druhé sady náhodně vybrali jeden výrobek. Jaká je pravděpodobnost, že to byl zmetek? \\[0.5cm]
  
  Označíme
  
  $A$: výrobek vybraný ze druhé sady byl zmetek
  
  $H_1$: výrobek přemístěný z první sady do druhé byl kvalitní
  
  $H_2$: výrobek přemístěný z první sady do druhé byl zmetek
  
  Máme
  $$
  P(A) = P(H_1)\cdot P(A|H_1) + P(H_2)\cdot P(A|H_2) = \frac{11}{12}\cdot \frac{1}{11} +\frac{1}{12}\cdot \frac{2}{11} = \frac{13}{132}.
  $$
  }
  
  
  

\frame {
  \frametitle{Bayesův vzorec}
    Nechť je dán úplný systém hypotéz $\{H_i, i\in I\}$. Potom pro jev $A$ s nenulovou pravděpodobností a pro libovolný jev $H_k$ platí tzv. I.~Bayesův vzorec
    $$
    P(H_k|A)= \frac{P(H_k)\cdot P(A|H_k)}{\sum\limits_{i\in I} P(H_i)\cdot P(A|H_i)}.
    $$
    
     
  }
  
  
  \frame {
  \frametitle{Bayesův vzorec -- příklad}
U jistého druhu elektrického spotřebiče se s pravděpodobností 0.1 vyskytuje výrobní vada. U spotřebiče s touto výrobní vadou dochází v záruční lhůtě k poruše s pravděpodobností 0.5. Výrobky, které tuto vadu nemají, se v záruční lhůtě porouchají s pravděpodobností 0.01. Jaká je pravděpodobnost, že výrobek, který se v záruční době porouchá, bude mít dotyčnou výrobní vadu?\\[0.5cm]

Označíme

$A$: výrobek se v záruční době porouchá\\
$H_1$: výrobek má výrobní vadu\\
$H_2$: výrobek nemá výrobní vadu    \\

$$
P(H_1|A)= \frac{P(H_1)\cdot P(A|H_1)}{P(H_1)\cdot P(A|H_1) + P(H_2)\cdot P(A|H_2)} = \frac{0.1\cdot 0.5}{0.059} = 0.847
$$
     
  }
  

\frame {
  \frametitle{Stochastická nezávislost}
  Řekneme, že jevy $A_1, A_2, \dots, A_n$ jsou {\bf stochasticky nezávislé}, právě když platí vztahy
  $$\forall\ i<j \quad P(A_i\cap A_j)=P(A_i)\cdot P(A_j)$$
  $$\forall\ i<j<k \quad P(A_i\cap A_j\cap A_k)=P(A_i)\cdot P(A_j)\cdot P(A_k)$$
  $$\vdots$$
  $$P(A_1\cap A_2\cap\dots\cap A_n)=P(A_1)\cdot P(A_2)\cdot \dots \cdot P(A_n)$$
  
  Stochastické nezávislosti se využívá velmi často. Např. máme-li stanovit pravděpodobnost nastoupení alespoň jednoho z jevů $A_1, A_2, \dots, A_n$, využijeme de Morganova pravidla:
  \small
  $$P(\bigcup_{i=1}^n A_i) = P(\overline{\bigcap_{i=1}^n \bar{A}_i}) = 1-P(\bigcap_{i=1}^n \bar{A}_i) = 1 - \prod_{i=1}^n P(\bar{A}_i) = 1 - \prod_{i=1}^n (1-P(A_i))
  $$
  \normalsize
  }
  
    
  \frame {
  \frametitle{Stochastická nezávislost - příklad}
  Pravděpodobnost, že semínko slunečnice vyklíčí, je 0.5. Zasejeme-li 7 semínek, jaká je pravděpodobnost, že alespoň jedno vyklíčí?
  
  Označíme
  $A_i$, $i=1, \dots, 7$: $i$-té semínko vyklíčí
  
  $A$: alespoň jedno semínko vyklíčí
  
  $$A=\bigcup_{i=1}^7 A_i$$
  
   $$P(A)=P(\bigcup_{i=1}^7 A_i)=\dots = 1 - \prod_{i=1}^7 (1-P(A_i)) = 1-\left(\frac{1}{2}\right)^7 = \frac{127}{128}$$
  }
  
  \frame {
  \frametitle{Opakované nezávislé pokusy}
  
  \begin{itemize}
  \item Provádíme jeden náhodný pokus. Výsledkem tohoto pokusu může být jen \enquote{úspěch} anebo \enquote{neúspěch}. Pravděpodobnost úspěchu bude $\theta$. Pokud označíme úspěch 1 a neúspěch 0, můžeme tuto situaci popsat vztahem tzv. {\bf alternativního rozdělení}
  $$ P(x)=\theta^x(1-\theta)^{1-x},\quad x\in\{0,1\}. $$
  \item Tento pokus budeme provádět $n$-krát nezávisle na sobě. Zajímat nás bude počet úspěchů $y=\sum_{i=1}^n x_i$, kde $x_i$ je výsledek $i$-tého alternativního pokusu. Pravděpodobnost nastoupení právě $y$ úspěchů z $n$ pokusů má tzv. {\bf binomické rozdělení} dané vztahem
  
  $$
  P_n(y)=\binom{n}{y} \theta^y (1-\theta)^{n-y}.
  $$
  
  \end{itemize}
  
  
  }
  
  \frame {
  \frametitle{Opakované nezávislé pokusy -- příklad}
  Pětkrát nezávisle na sobě hážeme třemi kostkami. Jaká je pravděpodobnost, že právě ve dvou hodech padnou tři jedničky?\\[1cm]
  
  Pravděpodobnost, že v jednom hodu třemi kostkami padnou tři jedničky je $\theta = \frac{1}{6^3} = \frac{1}{216}$.
  Dle vzorce binomického rozdělení máme
  $$
  P_5(2)=\binom{5}{2} \left(\frac{1}{216}\right)^2 \left(1-\frac{1}{216}\right)^{5-2}=
  $$
  $$
   = 10\cdot 0.00002143 \cdot 0.98617 = 0.0002.
  $$
  }
  
  
  
  
  \frame{\centerline{\bf Některé důsledky pro výpočet pravděpodobností}}

\frame{
  \frametitle{Podmíněná pravděpodobnost v různých situacích}
Předpokládejme, že $P(H) \neq 0$. Čemu je rovna pravděpodobnost $P(A|H)$, jsou-li jevy $A$; $H$\\[3mm]
\begin{itemize}
	\item[a)] stochasticky nezávislé
\end{itemize}
$$
P(A|H)=\frac{P(A\cap H)}{P(H)} = \frac{P(A)\cdot P(H)}{P(H)}=P(A)
$$
}



\frame{
  \frametitle{Podmíněná pravděpodobnost v různých situacích}
Předpokládejme, že $P(H) \neq 0$. Čemu je rovna pravděpodobnost $P(A|H)$, jsou-li jevy $A$; $H$\\[3mm]
\begin{itemize}
	\item[b)] neslučitelné
\end{itemize}
$$
P(A|H)=\frac{P(A\cap H)}{P(H)} = \frac{P(\emptyset)}{P(H)}=0
$$
}



\frame{
  \frametitle{Podmíněná pravděpodobnost v různých situacích}
Předpokládejme, že $P(H) \neq 0$. Čemu je rovna pravděpodobnost $P(A|H)$, jsou-li jevy $A$; $H$\\[3mm]
\begin{itemize}
	\item[c)] $H$ má za důsledek $A$
\end{itemize}
$$
P(A|H)=\frac{P(A\cap H)}{P(H)} = \frac{P(H)}{P(H)}=1
$$
}



\frame{
  \frametitle{Podmíněná pravděpodobnost v různých situacích}
Předpokládejme, že $P(H) \neq 0$. Čemu je rovna pravděpodobnost $P(A|H)$, jsou-li jevy $A$; $H$\\[3mm]
\begin{itemize}
	\item[d)] $A$ má za důsledek $H$
\end{itemize}
$$
P(A|H)=\frac{P(A\cap H)}{P(H)} = \frac{P(A)}{P(H)}
$$
}




\frame{
  \frametitle{Nastoupení alespoň jednoho z jevů}
Vyjádřete pravděpodobnost 
$$
P(\bigcup_{i=1}^n A_i)
$$
\begin{itemize}
	\item za předpokladu, že $A_1, \dots, A_n$ jsou neslučitelné
	$$
P(\bigcup_{i=1}^n A_i) = \sum_{i=1}^n P(A_i)
$$
	
	\onslide<2->
	\item za předpokladu, že $A_1, \dots, A_n$ jsou nezávislé
		$$
P(\bigcup_{i=1}^n A_i) = 1-\prod_{i=1}^n (1-P(A_i))
$$
\end{itemize}
}



\frame{
  \frametitle{Společné nastoupení jevů}
Vyjádřete pravděpodobnost 
$$
P(\bigcap_{i=1}^n A_i)
$$
\begin{itemize}
	\item za předpokladu, že $A_1, \dots, A_n$ jsou neslučitelné
	$$
P(\bigcap_{i=1}^n A_i) = 0
$$
	
	\onslide<2->
	\item za předpokladu, že $A_1, \dots, A_n$ jsou nezávislé
		$$
P(\bigcap_{i=1}^n A_i) = \prod_{i=1}^n P(A_i)
$$
\end{itemize}  }
  
  
  
  
  \frame{\centerline{\bf Aplikace pravděpodobnosti}}
  
  
  \frame {
  \frametitle{Příklad o plachetnicích}


Firma se rozhoduje, zda vstoupit na trh s novým typem malé sportovní plachetnice, což by se vyplatilo, kdyby ji aspoň 10 \% koupěschopných zákazníků koupilo. Určitému počtu náhodně vylosovaných zákazníků udělá firma fiktivní nabídku, v níž požaduje zatržení jedné ze čtyř možností: určitě bych koupil, zřejmě bych koupil, snad bych koupil, nekoupil bych. Přitom z~dřívějších zkušeností je známo, kolik procent zákazníků volících jednotlivé odpovědi skutečně přistoupí ke koupi.

}



\frame{
  \frametitle{Příklad o plachetnicích}

\begin{center}
\begin{tabular}{|l|r|r|}
\hline
odpověď& \% odpovědí& \% skutečných kupců\\
\hline
1. určitě bych koupil& 12 \% &40 \%\\
2. zřejmě bych koupil &23 \% &20 \%\\
3. snad bych koupil &17 \% &8 \%\\
4. nekoupil bych &48 \% &1 \%\\
\hline
\end{tabular}
\end{center}
\begin{itemize}
\item[a)] S jakou pravděpodobností koupí náhodně vybraný zákazník loď?
\item[b)] S jakou pravděpodobností ti, kteří skutečně koupili, předtím zaškrtli \enquote{nekoupil bych}.
\end{itemize}
}









\frame{
  \frametitle{Příklad o plachetnicích -- řešení}
  $A$ \dots náhodně vybraný zákazník koupí plachetnici\\[3mm]
  $H_i$ \dots náhodně vybraný zákazník patří do $i$-té skupiny, $i=1, \dots, 4$.
  \begin{center}
	\begin{tabular}{ccc}
	P($H_1$)=0,12  && P($A|H_1$)=0,4  \\
		P($H_2$)=0,23  && P($A|H_2$)=0,2  \\
			P($H_3$)=0,17  && P($A|H_3$)=0,08  \\
				P($H_4$)=0,48  && P($A|H_4$)=0,01  \\
\end{tabular}
\end{center}
  
  }
  
  
  \frame{
  \frametitle{Příklad o plachetnicích -- řešení}
  \begin{itemize}
	\item[a)] 
	\begin{eqnarray*}
	P(A)&=&\sum_{i=1}^4 P(H_i)P(A|H_i)\\
	&=&0,12\cdot 0,4 + 0,23\cdot 0,2+ 0,17\cdot 0,08 +0,48 \cdot0,01\\
	&=&0,1124 \textcolor{red}{\ > 0,10}
	\end{eqnarray*}
	Firmě se s novou plachetnicí vyplatí vstoupit na trh.
	\item[b)] 
	$$
	P(H_4|A)=\frac{P(H_4)P(A|H_4)}{P(A)} = \frac{0,01\cdot 0,48}{0,1124}=0,0427
	$$
	Ti zákazníci, kteří skutečně plachetnici koupili, předtím zaškrtli \enquote{nekoupil bych} s pravděpodobností 0,0427. 
\end{itemize}
  
  }



\frame{
  \frametitle{Zloději ve firmě}

Firma pokládá náhodně vybraným zaměstnancům otázku: \enquote{Odnesli jste si během minulého
roku některý z našich výrobků bez zaplacení?} Aby je ujistila o anonymitě dotazníku,
připojuje tuto instrukci: Hoďte si v soukromí mincí a padne-li líc, odpovězte bez ohledu
na skutečnost \enquote{ano}, jestliže padne rub odpovězte ve shodě se skutečností \enquote{ano}, nebo
\enquote{ne}.\\[5mm]
\begin{itemize}
\item[a)] Zvolte matematický model.
\item[b)] Vypočtěte pravděpodobnost odpovědi \enquote{ano} a pravděpodobnost vylosování zloděje.
\item[c)] Aproximujte pravděpodobnost vylosování zloděje, bylo-li dotázáno $n$ osob, z nichž $a$
osob odpovědělo \enquote{ano}.
\end{itemize}
}

\frame{
  \frametitle{Zloději ve firmě -- řešení}
\begin{itemize}
\item[a)] Zvolte matematický model.\\[5mm]
Náhodný pokus spočívá ve vylosování jedné osoby a v jednom hodu mincí.\\[2mm]
L \dots padl líc\\[2mm]
K \dots byl vylosován zloděj\\[2mm]
A \dots vylosovaný odpovědel \enquote{ano}
\end{itemize}
}

\frame{
  \frametitle{Zloději ve firmě -- řešení}
\begin{itemize}
\item[b)] Vypočtěte pravděpodobnost odpovědi \enquote{ano} a pravděpodobnost vylosování zloděje.\\[5mm]
\begin{eqnarray*}
P(A)&=& P(L\cup(L'\cap K))\\
&=&P(L) + P(L')\cdot P(K)\\
&=&\frac{1}{2} + \frac{1}{2}P(K)
\end{eqnarray*}

$$
P(K)=2P(A) -1
$$
\end{itemize}
}

\frame{
  \frametitle{Zloději ve firmě -- řešení}
\begin{itemize}
\item[c)] Aproximujte pravděpodobnost vylosování zloděje, bylo-li dotázáno $n$ osob, z nichž $a$
osob odpovědělo \enquote{ano}.\\[5mm]

$$
P(A)\approx \frac{a}{n}
$$\\[3mm]
$$
P(K)\approx 2\frac{a}{n} -1
$$
\end{itemize}
}



\frame{
  \frametitle{Antidrogový test}
Jistá společnost podrobuje uchazeče o zaměstnání antidrogovému testu. Uchazeči pochází
z oblasti, kde pouze půl procenta obyvatel užívá drogy. Citlivost užívaného testu je 99~\%
(tzn., že u 99~\% narkomanů test vyjde pozitivně) a dále test je specifický také na 99~\% 
(Tzn., že test vyjde negativně u 99~\% \enquote{ne-narkomanů}). Zdá se tedy, že test je relativně
přesný.\\[2mm] 

Určete pravděpodobnost, že osoba, jejíž výsledek testu je pozitivní, skutečně užívá
drogy.
}

\frame{
  \frametitle{Antidrogový test -- řešení}
P($A$) \dots u náhodně vybraného uchazeče test vyjde pozitivně\\[2mm]
P($H_1$) \dots uchazeč bere drogy\\[2mm]
P($H_2$) \dots uchazeč nebere drogy\\[2mm]

  \begin{center}
	\begin{tabular}{ccc}
	P($H_1$)=0,005  && P($A|H_1$)=0,99  \\
		P($H_2$)=0,995  && P($A|H_2$)=1-P($A'|H_2$)=0,01  \\
\end{tabular}
\end{center}
}


\frame{
  \frametitle{Antidrogový test -- řešení}
  \begin{eqnarray*}
P(A)&=&P(H_1)P(A|H_1)+P(H_2)P(A|H_2)\\
&=&0,005 \cdot 0,99 + 0,995 \cdot 0,01\\
&=&0,0149
\end{eqnarray*}

$$
P(H_1|A)=\frac{P(H_1)P(A|H_1)}{P(A)} = \frac{0,005\cdot 0,99}{0,0149}=0,3322
$$

Pravděpodobnost, že uchazeč, jemuž vyšel test pozitivně, skutečně bere drogy, je pouze 33,22 \%. Je tedy více pravděpodobné, že drogy nebere!\\[3mm]

Čím menší je pravděpodobnost zkoumaného jevu ($H_1$), tím větší je pravděpodobnost, že test bude \enquote{falešně} pozitivní. 

$$
P(A\cap H_1) = 0,00495 < P(A\cap H_2) = 0,00995
$$
}


\end{document}